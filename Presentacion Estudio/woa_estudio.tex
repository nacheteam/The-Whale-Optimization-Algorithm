\documentclass[10pt]{beamer}

\usepackage[utf8]{inputenc}
\usepackage[spanish, es-tabla]{babel}

\usetheme{metropolis}
\usepackage{appendixnumberbeamer}

\usepackage{booktabs}
\usepackage[scale=2]{ccicons}

\usepackage{pgfplots}
\usepgfplotslibrary{dateplot}

\usepackage{graphicx}

\usepackage{amsmath}
\usepackage{amsfonts}
\usepackage{amssymb}
\usepackage{amsthm}
\usepackage{esvect}

\usepackage{xspace}
\newcommand{\themename}{\textbf{\textsc{metropolis}}\xspace}

\title{The Whale Optimization Algorithm (WOA)}
\author{Ignacio Aguilera Martos}
\date{22 Junio 2018}
\institute{Metaheurísticas}

\begin{document}

\maketitle

\begin{frame}[fragile]{Contenidos}
  \setbeamertemplate{section in toc}[sections numbered]
  \tableofcontents[hideallsubsections]
\end{frame}

\section{Introducción del problema}

	\begin{frame}[fragile]{Competición CEC2014}
		\vspace{10px}
		\pause
		\metroset{block=fill}
		\begin{block}{WOA}
			\begin{itemize}
				\item Algoritmo bioinspirado en cómo cazan las ballenas jorobadas.
				\item Hecho por Seyedali Mirjalili y Andrew Lewis.
				\item Se ejecuta el algoritmo sobre las 20 primeras funciones de CEC2014.
			\end{itemize}
		\end{block}
		\begin{block}{CEC2014}
			\begin{itemize}
				\item Es una competición reconocida a nivel mundial.
				\item Se intenta resolver un problema de minimizacion con 30 funciones.
				\item En nuestro caso sólo lo hemos hecho con dimensión 10 y 30 aunque en la competición se hacía con dimensiones 10,30,50 y 100.
				\item El ganador de la competición fue L-SHADE.
			\end{itemize}
		\end{block}
	\end{frame}

\section{Descripción del algoritmo inicial}

	\begin{frame}[fragile]{Fases de la caza}
		\vspace{10px}
		\pause
		\metroset{block=fill}
		\begin{block}{Fases de la caza}
			\begin{itemize}
				\item Exploración para encontrar presas.
				\pause
				\item Caza de presas.
			\end{itemize}
		\end{block}
		\pause
		\begin{center}
			\includegraphics[scale=0.7]{./Imagenes/imagen1.jpg}
			\hspace{10px}
			\includegraphics[scale=0.22]{./Imagenes/imagen2.png}
		\end{center}
	\end{frame}

\section{Modelo matemático}

	\begin{frame}[fragile]{Aproximación a la presa}
		\includegraphics[scale=0.23]{./Imagenes/imagen3.png}
		\pause
		\vspace{5px}
		$D(t) = |\vec{C}\cdot \vec{X^*}(t)-\vec{X}(t)|$ \hspace{20px} $\vec{X}(t+1) = \vec{X^*}(t)-\vec{A}\cdot D(t)$ \\
		\pause
		$D'(t) = |\vec{X^*}(t)-\vec{X}(t)|$ \hspace{20px} $\vec{X}(t+1) = \vec{D'}(t)\cdot e^{bl}\cdot \cos (2\pi l)+\vec{X^*}(t)$ \\
		\vspace{10px}
		$\vec{A} = 2\cdot \vec{a}\cdot \vec{r}-\vec{a}$ \hspace{20px} $\vec{C}=2\cdot \vec{r}$ \\
		\pause
		Donde $X$ es la posición de la ballena, $X^*$ la posición de la presa, $\vec{r}$ un vector aleatorio con valores en el intervalo $[0,1]$ y $a\in [0,2]$ que se decrementa de forma lineal desde 2 hasta 0.
	\end{frame}
	
	\begin{frame}[fragile]{Ecuación real del movimiento}
		\pause
		$$
		\vec{X}(t+1)=
		\begin{cases}
		\vec{X}(t+1) = \vec{X^*}(t)-\vec{A}\cdot D(t) & si \ p<0.5\\
		\vec{X}(t+1) = \vec{D'}(t)\cdot e^{bl}\cdot \cos (2\pi l)+\vec{X^*}(t) & si \ p \geq 0.5
		\end{cases}
		$$
		\pause
		Donde p es un número aleatorio en el intervalo $[0,1]$ \\
		\pause
		En caso de no tener presa hacemos el movimiento lineal hacia un vector aleatorio.
	\end{frame}
	
	\begin{frame}[fragile]{Pseudocódigo}
		\includegraphics[scale=0.4]{./Imagenes/imagen4.png}
	\end{frame}

\section{Desarrollo de mejoras}

	\begin{frame}[fragile]{Desarrollo de mejoras}
		\vspace{10px}
		\pause
		\metroset{block=fill}
		\begin{block}{Fases}
			\begin{itemize}
				\item Inicialización aleatoria de las ballenas.
				\item Hacer una aproximación espiral a una solución aleatoria al principio del algoritmo.
				\item Incorporación de la búsqueda local Solis Wets.
				\item Incorporación de un esquema de Differential Evolution.
				\item Sustitución de Solis Wets por CMAES.
				\item Reemplazar la aproximación aleatoria para aproximarse a una posición aleatoria del espacio en vez de a una ballena aleatoria.
			\end{itemize}
		\end{block}
	\end{frame}

\section{Versión final}

	\begin{frame}[fragile]{Versión final}
		\includegraphics[scale=0.25]{./Imagenes/version_final_p1.png}
	\end{frame}

	\begin{frame}[fragile]{Versión final}
		\includegraphics[scale=0.25]{./Imagenes/version_final_p2.png}
	\end{frame}

\section{Resultados}

\section{Conclusiones}



\begin{frame}[standout]
	\huge Ideas y preguntas.
\end{frame}

\end{document}
